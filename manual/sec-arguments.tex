\section{Command Line Arguments}
\label{sec:arguments}

\subsection{Overview}

\subsubsection{Main Arguments}

\begin{tabular}{rllp{.6\textwidth}}
  \hline
  \texttt{-d} & \texttt{--datatype}   & \bf{nt},\bf{aa} & Data type is `nt' for nucleotide (default), `aa' for amino-acid sequences. \\
  \texttt{-i} & \texttt{--input}      & {\em filename} & Input MSA file in FASTA or PHYLIP format. Check section~\ref{sec:arg:input} \\
  \texttt{-o} & \texttt{--output}     & {\em filename} & Pipes the output into a file \\
  \texttt{-q} & \texttt{--partitions} & {\em filename} & Partitions filename in RAxML format. Check section~\ref{sec:arg:parts} \\
  \texttt{-p} & \texttt{--processes}  & {\em number\_of\_threads} & Number of concurrent threads \\
  \texttt{-r} & \texttt{--rngseed}    & {\em seed} & Sets the seed for the random number generator \\
  \texttt{-t} & \texttt{--topology}   & {\em topology\_type}. & Check section~\ref{sec:arg:topo} \\
               && \bf{ml}             & maximum likelihood \\
               && \bf{mp}             & maximum parsimony \\
               && \bf{fixed-ml-jc}    & fixed maximum likelihood (JC) \\
               && \bf{fixed-ml-gtr}   & fixed maximum likelihood (GTR) \\
               && \bf{fixed-mp}       & fixed maximum parsimony (default) \\
               && \bf{random}         & random generated tree \\
               && \bf{user}           & fixed user defined (requires -u argument) \\
  \texttt{-u} & \texttt{--utree}      & {\em filename} & User-defined tree in NEWICK format. Check section~\ref{sec:arg:topo}\\
              & \texttt{--force}      & & Forces output overriding \\
              & \multicolumn{2}{l}{\texttt{--disable-checkpoint}} & Disables checkpoint writing (skips generating the {\em .ckp} file) \\
  \hline
\end{tabular}

\subsubsection{Candidate Models}

  \begin{tabular}{rllp{.6\textwidth}}
    \hline
    \texttt{-a} & \texttt{--asc-bias} & algorithm[:values] & Includes ascertainment bias correction. Check section~\ref{sec:ascbias} for more details \\
                                     &&&  {\bf lewis}: Lewis (2001) \\
                                     &&&  {\bf felsenstein}: Felsenstein (requires number of invariant sites) \\
                                     &&&  {\bf stamatakis}: Leaché et al. (2015) (requires invariant sites composition) \\
     \texttt{-f} & \texttt{--frequencies} & [ef]           & Sets the candidate models frequencies \\
                                     &&& {\bf e}: Estimated - maximum likelihood (DNA) / empirical (AA) \\
                                     &&& {\bf f}: Fixed - equal (DNA) / model defined (AA) \\
     \texttt{-h} & \texttt{--model-het} & [uigfh] & Sets the candidate models rate heterogeneity \\
                                     &&& {\bf u}: *Uniform \\
                                     &&& {\bf i}: *Proportion of invariant sites (+I) \\
                                     &&& {\bf g}: *Discrite Gamma rate categories (+G) \\
                                     &&& {\bf f}: *Both +I and +G (+I+G) \\
                                     &&& {\bf r}: Free rate models (+R) \\
                                     &&& * {\it included by default} \\
     \texttt{-m} & \texttt{--models}  & {\em list} & Sets the candidate model matrices separated by commas \\
                                      && dna: & *JC *HKY *TrN *TPM1 *TPM2 *TPM3 *TIM1 *TIM2 *TIM3 *TVM *GTR \\
                                      && protein: & *DAYHOFF *LG *DCMUT *JTT *MTREV *WAG *RTREV *CPREV *VT *BLOSUM62 *MTMAM *MTART *MTZOA *PMB *HIVB *HIVW *JTT-DCMUT *FLU *STMTREV LG4M LG4X GTR \\
                                      &&& * {\it included by default} \\
    \texttt{-s} & \texttt{--schemes}  & {\em number\_of\_schemes} & Number of DNA substitution schemes. \\
                                      &&& {\bf 3}: JC, HKY, GTR \\
                                      &&& {\bf 5}: JC, HKY, TrN, TPM1, GTR \\
                                      &&& {\bf 7}: JC, HKY, TrN, TPM1, TIM1, TVM, GTR \\
                                      &&& {\bf 11}: All models defined in Sec~\ref{subsec:models} \\
                                      &&& {\bf 203}: All possible GTR submatrices \\
    \texttt{-T} & \texttt{--template} & {\em tool} & Sets candidate models according to a specified tool \\
                  && {\bf raxml}      & RAxML (DNA 3 schemes / AA full search) \\
                  && {\bf phyml}      & PhyML (DNA full search / 14 AA matrices) \\
                  && {\bf mrbayes}    & MrBayes (DNA 3 schemes / 8 AA matrices) \\
                  && {\bf paup}       & PAUP* (DNA full search / AA full search) \\
    \hline
  \end{tabular}

\subsubsection{Other options}

  \begin{tabular}{rllp{.6\textwidth}}
    \hline
      & \texttt{--eps} & {\em epsilon\_value}    & Sets the model optimization epsilon \\
      & \texttt{--tol} & {\em tolerance\_value}  & Sets the parameter optimization tolerance \\
      & \texttt{--smooth-frequencies}   & & Forces frequencies smoothing \\
   \texttt{-g} & \texttt{--gamma-rates}   & \bf{a},\bf{m} & Gamma rates per category are computed as `a' average (default), or `m' median. \\
   \texttt{-H} & \texttt{--no-compress} & & Disables pattern compression. \modeltest ignores if there are missing states \\
   \texttt{-k} & \texttt{--keep-params} & & Keep branch lengths fixed \\
   \texttt{-v} & \texttt{--verbose} & & Run in verbose mode \\
      & \texttt{--help}    & & Display this help message and exit \\
      & \texttt{--version} & & Output version information and exit \\
  \end{tabular}

\section{Model Optimization Settings}
\label{sec:optsettings}

\subsection{Input data}
\label{sec:arg:input}

The main and only required argument is the multiple sequence alignment file ($-i$ argument).
\modeltest supports PHYLIP and FASTA format.
All sequences must be alignned and have thus have the same sequence length.

PHYLIP format starts with a header line containing 2 integer values corresponding to the number of sequences and the sequence length.
The following lines are the individual taxa followed by the corresponding sequence.
Taxon names and sequences must \emph{not} contain whitespaces.
If that is the case in your alignment, please remove or replace every white space with any arbitrary character, such for example an underscore.

Please note that at this moment \modeltest does not support interleaved PHYLIP format.

{
\begin{verbatim}
TAXA_COUNT SEQ_LENGTH
TAXON_NAME_1 SEQUENCE_1
TAXON_NAME_2 SEQUENCE_2
TAXON_NAME_3 SEQUENCE_3
...
TAXON_NAME_N SEQUENCE_N
\end{verbatim}
}

Example:

{
\begin{verbatim}
5 20
taxon1 acgctatcgcgatcgatagc
taxon2 aaactagggcgatcgatagg
taxon3 acactatcg---tcgatagg
taxon4 acgctatcg---ccgatagg
taxon5 acgctaacgcgaacgttatc
\end{verbatim}
}

FASTA format does not contain any header, and it is formatted as a list of the sequences,
each of them covering 2 lines: the taxon name, and the sequence.

{
\begin{verbatim}
>TAXON_NAME_1
SEQUENCE_1
>TAXON_NAME_2
SEQUENCE_2
>TAXON_NAME_3
SEQUENCE_3
...
>TAXON_NAME_N
SEQUENCE_N
\end{verbatim}
}

The example below is analogous to the previous example in PHYLIP format:

{
\begin{verbatim}
>taxon1
acgctatcgcgatcgatagc
>taxon2
aaactagggcgatcgatagg
>taxon3
acactatcg---tcgatagg
>taxon4
acgctatcg---ccgatagg
>taxon5
acgctaacgcgaacgttatc
\end{verbatim}
}

\subsection{Models of evolution}
\label{subsec:models}

\modeltest implementes all 203 types of time-reversible substitution matrices,
with when combined with unequal/equal base frequencies,
gamma-distributed among-site rate variation and a proportion of invariable sites
makes a total of 1624 models.
Some of the models have received names:

\vspace{1em}
\begin{tabular}{l l l l l l}
\hline
{\bf Model} & {\bf Reference} & {\bf Free}   & {\bf Base}  & {\bf Substitution rates} & {\bf Substitution} \\
            &                 & {\bf param.} & {\bf freq.} &                          & {\bf code} \\
\hline
JC & \citep{Jukes-1969} & 0 & equal & AC=AG=AT=CG=CT=GT & 000000 \\
\hline
F81 & \citep{Felsenstein-1981} & 3 & unequal & AC=AG=AT=CG=CT=GT & 000000 \\
\hline
K80 & \citep{Kimura-1980} & 1 & equal & AC=AT=CG=GT;AG=GT & 010010 \\
\hline
HKY & \citep{Hasegawa-1985} & 4 & unequal & AC=AT=CG=GT;AG=GT & 010010 \\
\hline
TrNef & \citep{Tamura-1993} & 2 & equal & AC=AT=CG=GT;AG;GT & 010020 \\
\hline
TrN & \citep{Tamura-1993} & 5 & unequal & AC=AT=CG=GT;AG;GT & 010020 \\
\hline
TPM1 & =K81 \citep{Kimura-1981} & 2 & equal & AC=GT;AG=CT;AT=CG & 012210 \\
\hline
TPM1uf & \citep{Kimura-1981} & 5 & unequal & AC=GT;AG=CT;AT=CG & 012210 \\
\hline
TPM2 & & 2 & equal & AC=AT;CG=GT;AG=CT & 010212 \\
\hline
TPM2uf & & 5 & unequal & AC=AT;CG=GT;AG=CT & 010212 \\
\hline
TPM3 & & 2 & equal & AC=AT;AG=GT;AG=CT & 012012 \\
\hline
TPM3uf & & 5 & unequal & AC=CG;AT=GT;AG=CT & 012012 \\
\hline
TIM1 & \citep{Posada-2003} & 3 & equal & AC=GT;AT=CG;AG;CT & 012230 \\
\hline
TIM1uf & \citep{Posada-2003} & 6 & unequal & AC=GT;AT=CG;AG;CT & 012230 \\
\hline
TIM2 & & 3 & equal & AC=AT;CG=GT;AG;CT & 010232 \\
\hline
TIM2uf & & 6 & unequal & AC=AT;CG=GT;AG;CT & 010232 \\
\hline
TIM3 & & 3 & equal & AC=CG;AT=GT;AG;CT & 012032 \\
\hline
TIM3uf & & 6 & unequal & AC=CG;AT=GT;AG;CT & 012032 \\
\hline
TVMef & \citep{Posada-2003} & 4 & equal & AC;CG;AT;GT;AG=CT & 012314 \\
\hline
TVM & \citep{Posada-2003} & 7 & unequal & AC;CG;AT;GT;AG=CT & 012314 \\
\hline
SYM & \citep{Zharkikh-1994} & 5 & equal & AC;CG;AT;GT;AG;CT & 012345 \\
\hline
GTR & =REV \citep{Tavare-1986} & 8 & unequal & AC;CG;AT;GT;AG;CT & 012345 \\
\hline
\end{tabular}
\vspace{1em}

\modeltest includes the empirical amino acid matrices described in the table below.
If you expect a very long runtime according to the size of your data,
it is recommended to select {\em a priori} a sensible set of candidate matrices instead of evaluating all the available ones
(e.g., discarding those matrices estimated from different data).

\vspace{1em}
\begin{tabular}{ll}
  \hline
  {\bf Model} & {\bf Description} \\
  \hline
  Dayhoff &	General matrix~\citep{dayhoff1978} \\
  \hline
  JTT &	General matrix~\citep{jones1992} \\
  \hline
  DCMut/JTT-DCMut & Revised Dayhoff and JTT matrices~\citep{kosiol2005} \\
  \hline
  WAG & General matrix~\citep{whelan2001} \\
  \hline
  VT & General matrix~\citep{muller2000}  \\
  \hline
  cpREV & Chloroplast matrix~\citep{adachi2000} \\
  \hline
  rtREV & Retrovirus~\citep{dimmic2002} \\
  \hline
  stmtREV & Streptophyte mitochondrial land plants~\citep{liu2014} \\
  \hline
  mtArt & Mitochondrial Arthropoda~\citep{abascal2007} \\
  \hline
  mtMam & Mitochondrial Mammals~\citep{yang1998} \\
  \hline
  mtREV & Mitochondrial Verterbrate~\citep{adachi1996} \\
  \hline
  mtZoa & Mitochondrial Metazoa (Animals)~\citep{rota2009} \\
  \hline
  HIVb/HIVw & HIV matrices ~\citep{nickle2007} \\
  \hline
  LG & General matrix~\citep{le2008} \\
  \hline
  Blosum62 & BLOcks SUbstitution Matrix~\citep{henikoff1992} \\
  \hline
  PMB & Revised Blosum matrix~\citep{veerassamy2003} \\
  \hline
  FLU & Influenza virus~\citep{dang2010} \\
  \hline
  LG4M & 4-matrix mixture model with discrete $\Gamma$ rates ~\citep{le2012} \\
  \hline
  LG4X & 4-matrix mixture model with free rates~\citep{le2012} \\
  \hline
\end{tabular}

\subsection{Topology type}
\label{sec:arg:topo}

By default, \modeltest optimizes each single model using a fixed Maximum-Parsimony topology
with Maximum-Likelihood optimized branch lengths.
However, it allows other tree optimization techniques.
The topology type can be selected with $-t$ argument and it accepts the following values:

\begin{itemize}
  \item {\bf ml:} Optimize topology and branch lengths for each model
  \item {\bf fixed-ml-jc:} Build a ML topology with Jukes-Cantor model and fixes it for every other.
  \item {\bf fixed-ml-gtr:} Build a ML topology with GTR model and fixes it for every other.
  \item {\bf random:} Use a fixed randomly generated tree.
  \item {\bf user:} Use fixed user-defined topology
\end{itemize}

In addition to that, you can set a custom tree topology using $-u$ argument, followed by a file containing the tree in NEWICK format.
\modeltest works only with unrooted topologies, so if the user-defined tree is rooted, it will be automatically unrooted.
This argument is mandatory if the tree type was set to {\emph user}, and optional for ML trees.
In the latter case, the custom-defined tree is used as starting point for the ML optimization, while otherwise \modeltest uses a MP tree.

A {\emph random} tree topology can be interesting if one wants to measure how sensitive is the model selection process to the tree topology.
If you want to test several different random trees, do not forget to use different RNG seeds ($-r$ argument).


\subsection{Partitioning scheme}
\label{sec:arg:parts}

\modeltest is able to select individual models of evolution for each partition defined on the data set ($-q$ argument).
The partitioning scheme used may be defined in a file using RAxML-like format, where each partition is defined by one line in the file as follows:

{
\begin{verbatim}
DATA_TYPE, PARTITION_NAME = PARTITION_SITES
\end{verbatim}
}

Where:

\begin{itemize}
	\item {\bf DATA\_TYPE} can be {\emph DNA}/{\emph NT} or {\emph PROT}/{\emph AA}
  \item {\bf PARTITION\_NAME} is an arbitrary name for each partition
  \item {\bf PARTITION\_SITES} is the subset of sites that belong to the partition.
        They can be contiguous (e.g.,1-1000), or defined in several sections (e.g., $1-1000,2500-3000$).
        Additionally, one can specify a stride.
        For example, a partition covering all first codon positions in the first 1,000 sites is defined as $1-1000 \backslash 3$,
        second codon position is $2-1000 \backslash 3$, and third $3-1000 \backslash 3$.
        Second and third codon positions together would be $2-1000 \backslash 3,3-1000 \backslash 3$.
\end{itemize}

For example:

{
\begin{verbatim}
  DNA, GENE1 = 1-800
  DNA, GENE2_1 = 1701-2400\3
  DNA, GENE2_2 = 1702-2400\3
  DNA, GENE2_3 = 1703-2400\3
  DNA, GENE3 = 801-1700,2401-2500
\end{verbatim}
}

Partitions do not need to cover all sites in the MSA.
Every site which does not belong to any partition is just ignored.
Also, there must not be overlapping partitions (i.e., it is not allowed a site to belong to more than one partition).

\subsection{Ascertainment Bias Correction}
\label{sec:ascbias}

{\modeltest} incorporates 3 algorithms for including ascertainment bias correction in the candidate models.

Let $c$ be the sum of likelihoods ({\bf not} log-likelihoods) of the {\emph `dummy'}, or virtual invariant sites containing each of the states (eq.~\ref{eq:sumdummy}):,
$n$ is the number of sites,
$s$ is the number of states,
$\omega$ is the number of invariant sites,
and $\omega_i$ is the number of invariant sites for state $i$.

\begin{equation}
  \label{eq:sumdummy}
  c = \sum\limits_{i}^{s}L(s)
\end{equation}

\begin{itemize}
  \item Lewis (Lewis, 2001)
    \begin{equation}
      ln(L) = \sum\limits_{i}^{n}ln(L_i) - n \cdot ln(1-c)
    \end{equation}
  \item Felsenstein (Felsenstein, xx)
    \begin{equation}
      ln(L) = \sum\limits_{i}^{n}ln(L_i) + \omega \cdot ln(c)
    \end{equation}
  \item Stamatakis (Leach\'e et al. 2015)
    \begin{equation}
      ln(L) = \sum\limits_{i}^{n}ln(L_i) + \sum\limits_{j}^{s} \omega_j \cdot ln(L(j))
    \end{equation}
\end{itemize}

You can set ascertainment bias correction in {\modeltest} using the {\it -a} argument: {\emph -a algorithm[:values]},
where {\emph algorithm} must be {\emph lewis}, {\emph felsenstein} or {\emph stamatakis}.
Additionally, the weights of the dummy sites for Felsenstein's and Stamatakis' algorithms can be set using the {\emph value} optional argument.
For example:

\begin{itemize}
  \item Lewis' algorithm (no weights required)

  \texttt{\$ modeltest -i example-data/dna/aP6.fas -a lewis}

  \item Felsenstein's algorithm (sum of dummy sites weights required, values=${w_a + ... + w_t}$)

  \texttt{\$ modeltest -i example-data/dna/aP6.fas -a felsenstein:20}

  \item Stamatakis' algorithm (dummy sites weights required, values=$"w_a,w_c,w_g,w_t"$)

  \texttt{\$ modeltest -i example-data/dna/aP6.fas -a stamatakis:10,5,7,15}
\end{itemize}

The weights can also be set in the partitions file in a RAxML-like manner,
because if the analysis involves several partitions, the dummy sites weights are likely unequal.

There are 2 important conditions for using ascertainment bias correction:

\begin{enumerate}
\item The input alignment must {\emph not} contain invariant sites.
\item Models with a proportion of invariant sites (i.e., {\emph +I} and {\emph +I+G} must be excluded.
      If {\emph -h} argument for selecting the rate variation is present and it includes {\emph `g'} or {\emph `f'}, {\modeltest} will complain and stop.
\end{enumerate}

\subsection{Frequencies}
\label{sec:arg:freqs}

Nucleotide or amino acid stationary frequencies in a model of evolution can be either (i) defined {\em a-priori},
using fixed equal or empirical frequencies, or (ii) estimated from the data set at hand,
computing the empirical frequencies or estimating ML ones.
The latter involve $S-1$ additional degrees of freedom, where $S$ is the number of states (4 for DNA, 20 for protein data).

For nucleotide substitution models, \modeltest supports equal (no additional degrees of freedom) and ML frequencies (3 additional degrees of freedom).

For amino acid replacement models, \modeltest supports model-defined (no additional degrees of freedom) and empirical frequencies (19 additional degrees of freedom).

With $-f$ argument you can choose whether you want to include models with fixed and/or estimated frequencies using one of both options below.
By default, \modeltest behaves as including the argument $-f$ $ef$.

\vspace{1em}
\begin{tabular}{cll}
  {\bf Arg} & {\bf Nucleotide} & {\bf Amino acid} \\
  \hline
    $f$     & fixed equal      & fixed model \\
    $e$     & ML estimated     & empirical \\
\end{tabular}

% Fixed / Empirical / ML frequencies
% Frequency smoothing

\subsection{Per-site rate heterogeneity}
\label{sec:arg:ratehet}

With $-h$ argument you can choose whether you want to include models with per-site rate heterogeneity using one or more options below.
By default, \modeltest behaves as including the argument $-h$ $uigf$.

\vspace{1em}
\begin{tabular}{cl}
  {\bf Arg} & {\bf Rate heterogeneity model} \\
  \hline
    $u$     & No rate heterogeneity \\
    $i$     & proportion of invariant sites (+I) \\
    $g$     & discrete Gamma rates (+G) \\
    $f$     & both +I and +G together \\
\end{tabular}


\subsection{Substitution schemes}
\label{sec:arg:subschemes}

\subsection{Settings templates}
\label{sec:arg:templates}

In order to use the model of evolution selected by \modeltest in other phylogenetic inference tool,
you can select one of the settings templates such that you can make sure that the
candidate models set contains only models available in specific tools:

\begin{itemize}
  \item RAxML: JC/F81, K80/HKY and SYM/GTR models, with 4 gamma rate categories and a proportion of invariable sites.
  \item MrBayes: JC/F81, K80/HKY and SYM/GTR models, with 4 gamma rate categories and a proportion of invariable sites.
\end{itemize}

\subsection{Custom optimization thoroughness}
\label{sec:arg:thorough}

Thoroughness of the optimization process can be fine-tuned using 2 parameters:
a local tolerance parameter controls the convergence criteria for optimizing individual parameters,
and a global tolerance parameter decides whether to finish individual model optimization based on the log-likelihood score.
